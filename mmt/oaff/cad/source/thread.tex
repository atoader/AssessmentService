\begin{module}[id=thread]
\begin{definition}
  A \defii{screw}{thread}, is a helical ridge wrapped around (the inside or outside of a)
  cylinder or cone in the form of a helix, with the former being called a
  \defii{straight}{thread} (\defii{external}{thread} or \defii{internal}{thread}) and the
  latter called a \defii{tapered}{thread}.

  A straight thread is characterized by its
  \begin{enumerate}
  \item \defii{major}{diameter}, i.e. the largest diameter of the thread,
  \item \defii{minor}{diameter}, i.e. the largest diameter of the thread,
  \item \defi{angle}, i.e. the angle of the side of the ridge that makes up the thread,
  \item \defi{pitch}, i.e. the distance from one crest of the ridge to the next, and by
  \item the number of \adefi{starts}{start}, i.e. the number of ridges wrapped around the
    cylinder.
  \end{enumerate}
  The \defi{lead} of a thread is just its pitch times the starts.
\end{definition}
\end{module}

%%% Local Variables: 
%%% mode: latex
%%% TeX-master: "all"
%%% End: 

% LocalWords:  defii defi adefi
