\begin{module}[id=piping]
  \importmodule[load=\FCAD{flange/cds/physobj}]{physobj}

\begin{definition}
  A \defi{pipe} is a tubular section or hollow cylinder, usually but not necessarily of
  circular cross-section, used mainly to convey \trefi[physobj]{fluid}s.
\end{definition}

\begin{definition}
  A \defi{piping} (or \defii[piping]{piping}{system}) is a system of \trefi{pipe}s joined
  by welding, soldering, brazing, or via \trefi{fitting}s. Pipings are used to convey
  \trefi[physobj]{fluid}s (\trefi[physobj]{liquid}s and \atrefi[physobj]{gases}{gaseous})
  from one location to another.
\end{definition}

\begin{definition}
  A \defi{fitting} is used in pipings to connect straight pipe sections, to adapt to
  different sizes or shapes, and for other purposes, such as regulating or measuring
  \trefi[physobj]{fluid} flow.
\end{definition}

\begin{definition}
  A \defi{clean-out} (also called a \defii[clean-out]{pipe}{end}) is a particular fitting
  that allows access to the piping system for revision and
  cleaning. Clean-outs should be placed in accessible locations at regular intervals
  throughout a piping system. The minimum requirement is typically at the end of each
  branch in piping and at the base of each vertical stack.
\end{definition}
\end{module}
%%% Local Variables: 
%%% mode: latex
%%% TeX-master: "all"
%%% End: 

% LocalWords:  physobj defi trefi defii atrefi
