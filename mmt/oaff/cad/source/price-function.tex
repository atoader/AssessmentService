\begin{module}[id=price-function]
  \importmodule[load=\FCAD{flange/cds/financial-transaction}]{financial-transaction}
  \gimport{functions}

The \trefi[financial-transaction]{price} of an \trefi[financial-transaction]{offer} or a
\trefi[financial-transaction]{bid} may vary with the amount of the asset. Therefore it
is usually given as a function from quantities of goods to payments. 

\symtype{price-function}{\fntype\quantity\quantity}

\begin{definition}[functions=v]
  A \defii{price}{function} is a function $\fun{v}\quantity\quantity$, which assigns a
  \trefi[financial-transaction]{price} $v(q)$ of a quantity $q$ of an
  \trefi[financial-transaction]{asset}.
\end{definition}

The simplest \trefii{price}{function} is the following:

\symdef[name=unitprice]{unitpriceOp}{u_1}
\symdef{unitprice}[1]{\unitpriceOp}
\symdef[name=unitpricefunction]{unitpricefunctionOp}{p}
\symdef{unitpricefunction}[1]{\prefix\unitpricefunctionOp{#1}}
\begin{definition}
  The \defiii{unit-based}{price}{function} $\fun\unitpricefunctionOp\quantity\quantity$ with \defii{unit}{price} $\unitprice{\livar{q}1}$ for \defii{unit}{quantity}
  $\livar{q}1$ is given as
  $\defeq{\unitpricefunction{\livar{q}1}}{\scalarqmul{\scalarqdiv{q}{\livar{q}1}}{\unitprice{\livar{q}1}}}$
\end{definition}

Often a \trefii{price}{function} comes in the form of a unit-based price function with a
discount, which is given on a unit price for any purchase of a quantity that is above a
given threshold.

\symdef{discountedprice}[3]{\prefix{{#1}_{#2}}{#3}}
\begin{definition}[functions={d,p}]
  A \defi{discount} is a function $\fun{d}\quantity\RealNumbers$ from (asset) quantities
  to real numbers. Given a \trefii{price}{function} $p$, it induces the
  \defiii{discounted}{price}{function}
  $\defeq{\discountedprice{p}dq}{\scalarqmul{d(q)}{p(q)}}$.
\end{definition}

Instead of discounting a whole price function, we can discount the unit price. 

\symdef[name=discountedunitprice]{discountedunitpriceOp}[2]{u_{#1}^{#2}}
\symdef{discountedunitprice}[3]{\prefix{\discountedunitpriceOp{#1}{#2}}{#3}}
\begin{definition}[id=discounted-unit-price.def,functions=d]
  Given a \trefi{discount} $d$ and a unit price $\unitprice{q}$, the
  \defiii{discounted}{unit}{price} is the function
  $\funsuchthat{\discountedunitpriceOp{d}{\unitprice{q}}}\quantity\quantity{q}{\scalarqmul{d(q)}{\unitprice{q}}}$.
\end{definition}

\begin{assertion}[type=observation]
  Given a \trefi{discount} $d$ and a unit price $\unitprice{q}$, the
  \trefiii{discounted}{price}{function} induced by the
  \trefiii{unit-based}{price}{function} is (the same as) the
  \trefiii{unit-based}{price}{function} given by the \trefiii{discounted}{unit}{price}.
\end{assertion}

 Usually, a discount comes as a step function.

\begin{definition}[functions=d]
  If a \trefi{discount} $d$ is a left step function, i.e.  $d(q)=c$ for all
  $\qbetween{q}{\livar{q}0}{\livar{q}1}$ where $\livar{q}0$ and $\livar{q}1$ are adjacent
  step points. Then we call $\livar{q}0$ the \defii{minimum}{purchase} for the price $c$.
\end{definition}

In cases where a potential buyer has more than one \trefi[financial-transaction]{offer} (e.g. by multiple vendors) we use
a \trefiii{price}{comparison}{function}:

\begin{definition}[functions={c,p}]
  A \defiii{price}{comparison}{function} is a function
  $\fun{c}{\legalperson,\quantity}\quantity$ such that $c(l,q)=p(q)$, where $p$ is the
  \trefii{price}{function} of the current \trefi[financial-transaction]{offer} from vendor $l$.
\end{definition}


\end{module}

%%% Local Variables: 
%%% mode: latex
%%% TeX-master: "all"
%%% End: 

% LocalWords:  winograd legalentities defii trefii defi defiii ednote dmath dq
% LocalWords:  KWARCslides trefi trefi symdef unitpricefunction symtype fntype
% LocalWords:  unitpricefunctionOp defeq scalarqmul scalarqdiv unitprice
% LocalWords:  unitpriceOp discountedprice discountedunitprice funsuchthat
% LocalWords:  discountedunitpriceOp trefiii qbetween legalperson
