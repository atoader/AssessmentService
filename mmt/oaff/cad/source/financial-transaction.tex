\begin{module}[id=financial-transaction]
  \importmodule[load=\SiSsI{winograd/cds/legalentities}]{legalentities}
  \importmodule[load=\KWARCslides{units/en/quantities}]{quantities}
\begin{definition}
  A \defii{financial}{transaction} is an agreement, communication, or movement carried out
  between two \trefii[legalentities]{legal}{person}s to exchange an \defi{asset} (goods or
  services) for \defi{payment} (usually a quantity of money called the
  \defii{transaction}{price} or \defii[transaction-price]{traded}{price}). The
  \trefii[legalentities]{legal}{person} providing the asset in a transaction is called the
  \defi{seller} (or \defi[seller]{vendor}) and the \trefii[legalentities]{legal}{person}
  providing the payment the \defi{buyer}.
\end{definition}

A financial transaction is usually initiated by either an \trefi{offer} or a \trefi{bid},
which (after negotiations) leads to a transaction.  Therefore we distinguish the
\trefii{asking}{price} (the amount asked by the \atrefi{vendor}{seller}) from the
\trefii{bid}{price} (the quantity of payment offered by a \trefi{buyer}), which may differ
from the \trefii{transaction}{price}.

\begin{definition}[id=bid.def]
  A \defi{bid} is a communication in which a potential \trefi{buyer} offers to buy a
  certain \trefi{asset} for a given \trefi{payment}, the \defii{bid}{price}.
\end{definition}

\begin{definition}[id=offer.def]
  An \defi{offer} is a communication in which a potential \trefi{seller} offers to buy a
  certain \trefi{asset} for a given \trefi{payment}, the \defii{asking}{price}
\end{definition}

\begin{definition}
  If a \trefi{financial}{transaction} contains more than one position, then we distinguish
  the \defii{individual}{price} of a position from the \defii{overall}{price} of the
  transaction, which often the sum of all individual prices.
\end{definition}

\begin{definition}
  If we do not want to distinguish between \trefii{bid}{price}, \trefii{offer}{price},
  \trefii{transaction}{price}, \trefii{individua}{price}, and \trefii{overall}{price} we
  simply use the word \defi{price} to mean any of them.
\end{definition}

\end{module}

%%% Local Variables: 
%%% mode: latex
%%% TeX-master: "all"
%%% End: 

% LocalWords:  winograd legalentities defii trefii defi defiii ednote dmath
% LocalWords:  timeinterval KWARCslides trefi trefi symdef unitpricefunction
% LocalWords:  unitpricefunctionOp defeq scalarqmul scalarqdiv atrefi
