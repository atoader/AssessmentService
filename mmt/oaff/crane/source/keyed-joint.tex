\begin{module}[id=keyed-joint]
\importmodule[load=\FCAD{crane/cds/torque}]{torque}
  \begin{definition}
    A \defi{key} is a machine element used to connect a rotating machine element to a
    shaft. The key prevents relative rotation between the two parts and enables
    \trefi[torque]{torque} transmission. A key is used for temporary fastening. For a key
    to function, the shaft and rotating machine element must have a \defi{keyway}, also
    known as a \defi[keyway]{keyseat}, which is a slot or pocket the key fits in. The
    whole system is called a \defii{keyed}{joint}. A \trefi{keyed}{joint} still allows
    relative axial movement between the parts.
  \end{definition}

  \begin{definition}
    In a \defiii{feather}{key}{joint}, both the key and the rotating machine element have
    rectangular \trefi{keyway}s between which a rectangular \trefi{key} is placed for
    joining.
  \end{definition}
\end{module}
%%% Local Variables: 
%%% mode: latex
%%% TeX-master: "all"
%%% End: 
