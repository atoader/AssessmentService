\begin{module}[id=torque]
  \gimport{crossproduct}
  \gimport{norm}
  \importmodule[load=\KWARCslides{missing/en/arithmetics}]{arithmetics}
  \symdef{torque}{\tau}
  \symvariant{torque}{moment}{M}
  \begin{definition}
    The \defi{torque} $\torque$, \defi[torque]{moment} or \defii[torque]{moment
      of}{force}, is defined to be the rate of change of angular momentum of an
    object. When it is called moment, it is commonly denoted as
    \notatiendum{$\torque[moment]$}.
  \end{definition}

  \begin{omtext}
    The magnitude of torque depends on three quantities: the force applied, the length of
    the lever arm connecting the axis to the point of force application, and the angle
    between the force vector and the lever arm
    \[\torque=\crossproduct{r}F\qquad\text{or}\qquad
    \norm\torque=\atimes{\norm{r},\norm{F},\sin\theta}
    \]
    where $\torque$ is is the torque vector and $\norm\torque$ is the magnitude of the
    torque, $r$ is the displacement vector (a vector from the point from which torque is
    measured to the point where force is applied), $F$ is the force vector, and $\theta$
    is the angle between the force vector and the lever arm vector.
  \end{omtext}
\end{module}
%%% Local Variables: 
%%% mode: latex
%%% TeX-master: "all"
%%% End: 
