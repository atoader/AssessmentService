\documentclass[a4paper]{omdoc}
\usepackage{stex,stc-smglom}
\usepackage{ngerman}
\usepackage{amstext}
\usepackage{hyperref}
\usepackage{ed}
\input{../../../lib/paths}
\title{Construction Assignment 1}
\author{Prof. Dr.-Ing. Sandro Wartzack}

\begin{document}
\maketitle
\usemodule[load=\KWARCslides{units/en/time}]{time}
\usemodule[load=\KWARCslides{units/en/length}]{length}
\usemodule[load=\KWARCslides{missing/en/arithmetics}]{arithmetics}
\usemodule[load=\FCAD{crane/cds/crane}]{crane}
\usemodule[load=\FCAD{crane/cds/welding}]{welding}
\begin{omgroup}{Task Description}
\begin{omtext}
  The aim of this assisgnment is to contruct an \trefii[crane]{assembly}{crane} for lifting
  heavy building elements and machine components in workshops. \textbf{The principle
    sketch below is solely intended for orientation and clarification of the boundary
    conditions of the construction assignments!}
\end{omtext}

\begin{figure}[ht]
\mycgraphics[width=8cm]{\FCAD{crane/spec/PIC/principle-solution-crane}}
\caption{Concept of an Assembly Crane}\label{fig:principle-solution-crane}
\end{figure}

\begin{omtext}
  The \trefii[crane]{assembly}{crane} to be constructed can be divided into two principal
  groups: The \trefii[foo]{base}{frame} (1) and the \atrefii[winch]{winch unit}{winch}
  (2).
\end{omtext}


\begin{omtext}
  The \trefi[crane]{base}{frame} (1) supports all components of the
  \trefi[crane]{assembly}{crane} and transfers forces induced by the the load to be
  floor. To reduce costs and simplify provisioning, the assembly crane should be realized
  as a \trefii[welding]{weldment}{design}, ideally from normed
  \trefi[foo]{steel}{profile}s. The design should take the intended load into account
  appropriately. The \trefi[crane]{base}{frame} should be designed to ensure a
  sufficiently high strength and rigidity under maximal load $F_{max}$. Moreover the
  \trefi[crane]{base}{frame} must provide convenient interfaces for the additional
  components of the \trefi[crane]{crane}. To ensure the centered positioning of
  construction elements, the lower part of the \trefi[crane]{base}{frame} should be
  realized from two parallel \trefi[foo]{profiles}. The \trefi[foo]{clearance} between the
  two legs of the base frame should not fall below $b_{min}$ and the overall width should
  not exceed $b_{max}$. To enable transport of loads, the \trefi[crane]{base}{frame} is to
  be equipped with sufficiently strong \trefi[castor]{castors}:
  \trefii[castor]{rigid}{castor}s on the front and \trefii[castor]{swivel}{castor}s on the
  back of the \trefii[crane]{base}{frame}. To ease handling, the frame should contain a
  subtable handle bar.
\end{omtext}

\begin{omtext}
  The \trefi[hoist]{hoist} consists of a (also to be constructed) electically powered
  \trefi[winch]{winch} (2), that is connected to a \trefi[hoist]{lifting}{hook} (4) via a
  \trefi[cable]{cable} that is passed through some \trefi[sheave]{scheave}s. This enables
  continuous lifting, lowering, and holding of loads.
\end{omtext}

\begin{figure}[ht]
\mycgraphics[width=7cm]{\FCAD{crane/spec/PIC/principle-solution-winch}}
\caption{Prinzipdarstellung der Windeneinheit (2)}\label{fig:principle-solution-winch}
\end{figure}

\begin{omtext}
  The \trefi[winch]{winch} unit (2) itself (see
  Figure~\ref{fig:principle-solutions-winch}) consists of a drive motor with
  \atrefi[worm-drive]{worm geared}{worm}{drive} (5)\ednote{MK: translate ``in
    Flanschausf"uhrung''}, a \trefi[winch]{spool} (6), and an electrically driven disc
  brake (9). To make moving the \trefi[crane]{assembly}{crane} easy, the frame of the
  winch unit and the \trefi[winch]{spool} are to be realizeed as light-weight sheet metal
  \trefi[welding]{weldment}{design}s. The \trefi[winch]{spool} consists of wound-up and
  lenghwise \trefi[welding]{welded} sheet metal drum (6a) together with the two rigid drum
  sides (6b). The \trefi[torque]{torque} of the \trefi[motor]{motor} is transferred to the
  axis of the \trefi[winch]{spool} via a \trefiii[keyed-joint]{feather}{key}{joint}. The
  axis is rigidly welded to the drum sides and is itself mounted in
  \begin{oldpart}{MK: still needs to be translated} a Fest-Los-Lageranordnung (8)
    w"alzgelagert. Die als Halte- und Sicherheitsbremse ausgelegte Lamellenbremse (9) ist
    mittels eines Kegelpressverbands (10) fliegend auf der Seiltrommelwelle befestigt.
  \end{oldpart}
\end{omtext}

\begin{omtext}
  The whole \trefi[winch]{winch} unit is mounted on a frame construction (11) that is
  fastened to the crane frame with a through-bold connection. The design of the winch
  mount, has to allow that components can be mounted.
\end{omtext}

\begin{omtext}
  Werkstoffauswahl, Oberfl"acheng"ute und Ausf"uhrung aller bearbeiteten Fl"achen richten
  sich nach Funktion und Beanspruchung der jeweiligen Bauteile bzw. Kontaktfl"achen und
  sind unter Ber"ucksichtigung einer m"oglichst kosteng"unstigen Herstellung sinnvoll
  festzulegen.
\end{omtext}

\begin{omtext}
  Gegeben sind die folgenden Kenndaten und Hauptabmessungen:
\end{omtext}

\begin{omtext}[title=Abmessungen / Lasten]\strut\\
\begin{tabular}{ll}
Maximal load	 & $F_{max} = 15 kN$\\
Maximal height between floor and lower fib edge& 	$h_{max} = \quantityof{1900}\SImillimetre$\\
wheel displacement& $l_{Rad} = \quantityof{1800}\SImillimetre$\\
Fib length  & 	$l_{Ausleger} = \quantityof{1300}\SImillimetre$\\
Minimal clearance between base frame legs & 	$b_{min} = \quantityof{700}\SImillimetre$\\
Maximal width & $b_{max} = \quantityof{900}\SImillimetre$
\end{tabular}
\end{omtext}

\begin{omtext}[title=Werkstoffkennwerte]\strut\\
\begin{tabular}{ll}
Material Motor Axis& 	C22E\\
Material Brake Axis& C45E
\end{tabular}
\end{omtext}

\begin{omtext}[title=Beiwerte Seil / Trommel]\strut\\
\begin{tabular}{ll}
Beiwert $c$ nach DIN 15020 & $c = 0,08$ \\
Beiwert $h_1$ nach DIN 15020 & $h_1 = 14$\\
Beiwert $h_2$ nach DIN 15020 & $h_2 = 1$\\
Hublastbeiwert  & $\psi = 1,12$
\end{tabular}
\end{omtext}

\begin{omtext}[title=Angaben zu Zukaufteilen]\strut\\
\begin{tabular}{lp{5cm}}
Getriebemotor & SEW SAF 77R37 DRS 80M4\\
Ausgangsdrehzahl Getriebemotor & $n_a = \quantityof{6,4}{\power\SIminute{-1}}$\\
Kranhaken & Gabelkopf- oder "Osenhaken 	nach DIN EN 1677\\
Bremse	& Elektromagnetische Lamellen-
		bremse der Fa. Stromag, Typ EMB 160 in Sonderausf"uhrung:
                vgl. 4 – sonstige Hinweise
\end{tabular}
\end{omtext}

\begin{omtext}[title=Gebrauch und Produktion]\strut\\
\begin{tabular}{ll}
Einsatzdauer & 15 Jahre\\
Arbeitsspiele	& 20 H"ube/Arbeitstag\\
Geplantes Produktionsvolumen& 100 St"uck/Jahr
\end{tabular}
\end{omtext}
\end{omgroup}
\begin{appendix}
\begin{omgroup}{Hilfestellungen}
\begin{omgroup}{Auslegung von Seiltrommeln}
\begin{omtext}[title=Seiltrommel einlagig bewickelt]
  Erforderliche Windungszahl zum Aufwickeln der Seill"ange $L_S$ bei einem
  Wickeldurchmesser $D$:

\[W=2+\frac{L_S}{\atimes[cdot]{D,\pi}}\]

Hiermit l"asst sich f"ur den Seildurchmesser $d$ die erforderliche L"ange $L$ der
Seiltrommel zur Aufwicklung des Seils absch"atzen, wobei $p$ der Seilsteigung entspricht:

\[L=\atimes[cdot]{W,p}+d\]
\end{omtext}

\begin{omtext}[title=Hinweis] 
  F"ur nicht gerillte Seiltrommeln entspricht $p$ dem Durchmesser des Seils.
\end{omtext}

\begin{omtext}[title=Trommelwanddicke]
  Beim Aufwickeln des belasteten Seils schn"urt sich der Trommelzylinder ein. Nach
  \textsc{Ernst} sind die in einem Mantelringsegment tangential wirkende Druckspannung
  $\sigma_d$ und die durch Einschn"urung des Trommelzylinders in Achsrichtung wirkende
  Biegespannung $\sigma_b$ f"ur die Bemessung der kleinsten Trommelwanddicke $t_m$ im
  Bereich der ersten auflaufenden Seilwindung ma"sgebend:

  \[\sigma_d=-\frac{\atimes[cdot]{0.5,F_S,\psi}}{\atimes[cdot]{p,t_m}}
  \qquad\text{und}\qquad
  \sigma_b=\pm\frac{\atimes[cdot]{0.96,F_S,\psi}}{\sqrt{\atimes[cdot]{D,\power{t_m}3}}}
  \]

  mit $F_SS$ Seilzugkraft, $\psi$ Hublastbeiwert, $D$ Trommeldurchmesser, $p$
  Seilsteigung.
\end{omtext}

\begin{omtext}
  Nach Berechnung von $\sigma_d$ und $\sigma_b$ ist nachzuweisen, dass:

\[\sigma_\nu=\sqrt{\power{\sigma_d}2+\power{\sigma_b}2 - \sigma_d\sigma_b}\leq\sigma_{zul}\]

Bei Trommeln l"anger als $1,5D$ ist die Beulsicherheit zu "uberpr"ufen!
\end{omtext}

\begin{omtext}
  Die Torsionsschubsspannung $\tau_t$ infolge des zu "ubertragenden Drehmomentes kann auf-
  grund des verh"altnism"a"sig gro"sen Trommeldurchmessers vernachl"assigt werden, ebenso
  bei relativ gedrungenen Trommeln die Biegebeanspruchung $\sigma'_b$ durch den Zug des
  ablaufenden Seils.
\end{omtext}

\begin{omtext}[title=Quelle]
  Dubbel, Taschenbuch f"ur den Maschinenbau 19. Auflage
\end{omtext}
\end{omgroup}
\end{omgroup}
\end{appendix}
\end{document}

%%% Local Variables: 
%%% mode: LaTeX
%%% TeX-master: t
%%% End: 


