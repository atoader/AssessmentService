\begin{module}[id=castor]
  \importmodule[load=\KWARCslides{units/en/angles}]{angles}
  \begin{definition}
    A \defi{caster} (or \defi[caster]{castor}) is an undriven, single, double, or compound
    \trefi[wheel]{wheel} mounted on a \defi{fork} that is designed to be mounted to the
    bottom of a larger object (the ``vehicle'') so as to enable that object to be easily
    moved. They are available in various sizes, and are commonly made of rubber, plastic,
    nylon, aluminum, or stainless steel.

    There are two kinds of \trefi{caster}s:
  \begin{itemize}
  \item \defii{rigid}{caster}s whose fork is fixed relative to the vehicle, and 
  \item \defii{swivel}{caster}s have an additional swivel joint above the fork allows the
    fork to freely rotate about $\quantityof{360}{\arcdegree}$, thus enabling the wheel to
    roll in any direction.
\end{itemize}
\end{definition}

\begin{omtext}
  Castors are found in numerous applications, including shopping carts, office chairs, and
  material handling equipment. High capacity, heavy duty casters are used in many
  industrial applications, such as platform trucks, carts, assemblies, and tow lines in
  plants. Generally, casters operate well on smooth and flat surfaces.
\end{omtext}
\end{module}
%%% Local Variables: 
%%% mode: latex
%%% TeX-master: "all"
%%% End: 
