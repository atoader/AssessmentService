Besides being an API, the main interface for both humans and machines to the {\mmt} system is via the web server. See Sect.~\ref{sec:shell:server} for how to start and stop the server.

It does not matter whether a request originates from the local or a remote machine.

\subsection{Human Interface}

The {\mmt} web server can be accessed with any browser, e.g., by pointing it to \code{http://localhost:8080} after starting the server with \code{server on 8080} on the shell.

\subsection{Machine Interface}

The server responds to a number of requests that permit software systems to interact with {\mmt}.

\paragraph{GET requests}
A GET request of the path \code{/:mmt?URI\_A} is answered with the result of \code{URI A} where \code{A} is any action taking a MMT-URI. Spaces in \code{A} must be written as underscores, special characters in \code{A} must be \%-encoded.
If the URI contains less than $3$ $?$, the missing components default to being empty.

\begin{example}
\url{http://localhost:8080/:mmt?D?Q?R?component_type_present_U}
retrieves the type of \code{D?Q?R} rendered with style \code{U}.
\end{example}

\paragraph{POST requests}
POST requests are used to access the query \cite{rabe:querying:12} and the computation server.\ednote{add details}